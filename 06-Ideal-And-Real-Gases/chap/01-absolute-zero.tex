\chapter{Absolute Zero}

The temperature of absolute zero is determined by measuring thermal expansion of a gas and extrapolating to the point of zero pressure.

According to the ideal gas law
\begin{equation}\label{eq:ideal-gas}
	pV = nRT,
\end{equation}
pressure $p$ is proportional to (absolute) temperature $T$.
To relate absolute temperature to the celsius scale, which differs by a constant temperature offset $T_\text{o}$: $T_\text{C} \, (\si{\celsius}) = T \, (\si{\kelvin}) + T_\text{o}$, a modified version of \autoref{eq:ideal-gas},
\begin{equation*}
	p(T) = p_0 \cdot \left(1 + \alpha \, T_\text{C}\right),
\end{equation*}
is used.
The point of zero pressure in the units of $T_\text{c}$ is $-T_\text{o} = -\frac{1}{\alpha}$.

As $p(T)$ is purely linear, two points $(p, T)$ are sufficient to calculate $\alpha$.
In the experiment two easily reproducible temperatures, the melting and boiling point of water, are used.

To correct for the temperature dependent boiling point of water, the ambient pressure is measured:
\begin{equation}
	p_\text{A} = \SI{996}{\milli\bar} \quad \rightarrow \quad T_\mathrm{b, H_2O} = \SI{99.49}{\celsius}\footnote{\url{http://www.wolframalpha.com/input/?i=water+boiling+point+996mBar}}
\end{equation}

$\alpha_1$ is calculated from the two data points as
\begin{equation*}
	\alpha_1 = \frac{p_\text{b} - p_0}{p_0 \cdot T_\text{b}} = \SI{3.55e-3}{\per\celsius},
\end{equation*}
with the pressure $p_0$ at \SI{0}{\celsius} and the pressure $p_\text{b}$ and temperature $T_\text{b}$ (in \si{\celsius}) at the boiling point of water.

The effects of thermal expansion of the test vessel are corrected:
\begin{equation*}	%http://www.wolframalpha.com/input/?i=264mmHg%2F(996mBar*99.45%C2%B0C)+%2B+2.5e-5%2Fcelsius+*+(231mmHg+%2B+996mBar)%2F996mBar
	\alpha = \alpha_1 + \frac{p_\text{b}}{p_0} \cdot \gamma = \SI{3.59e-3}{\per\celsius},
\end{equation*}
which corresponds to a zero point of \SI{-278.9}{\celsius}.

\begin{table}
	\centering
	\caption{\textbf{relative air pressure} in test vessel with fixed volume at various temperatures, pressure relative to $p_\text{A} = \SI{996}{\milli\bar}$}
	\begin{tabular}{cS}
		\toprule
		condition&
		{relative pressure (\si{\mmHg})}\\
		\midrule
		ambient (\SI{27.6}{\celsius})&	42\\
		freezing&	-33\\
		boiling&	231\\
		\bottomrule
	\end{tabular}
\end{table}

\begin{equation*}
	\alpha_1 = \num{9.4703e-4}\\
	\alpha = \num{9.8087e-4}\\
	T_0 =
\end{equation*}
