\chapter{Adiabatic Exponent Of Air}
This experiment is used to determine the adiabatic exponent of air, based on \texttt{Rüchardt}'s method.

\section{Theory}
A glass piston is placed on top of a gas volume \todo{held captive by a terroristic glass bottle}.
The piston is excited to vibrations.
As the frequency of these vibrations tend to be too high to be counted by eye, a combination of a fixed coil, a permanent magnet mounted to the piston and a frequency counter is used to measure the oscillation frequency.

The periodic expansion/compression of this gas is approximately adiabatic. It holds
\begin{align}
	&pV^{\kappa} &&= \text{const.}& \nonumber \\
	\Leftrightarrow\ &\d p &&= -\kappa\cdot p\frac{\d V}{V}&&|\cdot A \nonumber \\
	\Leftrightarrow\ &\d F &&= -\kappa A^2\cdot\frac{p}{V}\cdot\d x &&\text{(using $A\cdot \d p=\d F$ and $A\cdot\d x=\d V$).} \label{eq:adiabatic_df}
\end{align}\todo{fix shitty left alignment of align}

\autoref{eq:adiabatic_df} is a linear relation with a resonant frequency of
\begin{align*}
	f&=\frac{1}{2\pi}\sqrt{\frac{k}{m}},
\end{align*}
where $k=\kappa A^2\cdot\sfrac{p}{V}$.
This results in
\begin{align}\label{eq:kappa}
	\kappa&=\left(\frac{2\pi f}{A}\right)^2\cdot\frac{mV}{p},
\end{align}
where\par
\begin{tabular}{ll}
	$f-$	&	frequency of oscillation \\
	$m-$	&	cylinder mass (including magnet) \\
	$A-$	&	cross section \\
	$V-$	&	gas volume \\
	$p-$	&	gas pressure (accounting for magnet mass)\\
\end{tabular}\par
denote the characteristic properties of the setup.

\section{Errors}
Systematic errors on the mass $m$ and the diameter of the piston $d$ are introduced as
\begin{gather*}
	m=\SI{109.80(5)}{\gram} \\
	d=\SI{31.1(1)}{\milli\meter}\rightarrow A=\SI{759.6}{\milli\meter\squared}.
\end{gather*}
The effect of the diameter $d$'s error on the cross section $A$ is calculated as
\begin{equation*}
	\Delta A=A\cdot2\frac{\Delta d}{d}=\SI{4.9}{\milli\meter\squared}.
\end{equation*}

As these errors are not correlated, \texttt{Gauss}' error propagation is used to determine the error on $\kappa$.
It holds
\begin{align}
	\Delta\kappa&=\sqrt{\left(\pdb{\kappa}{m}\cdot\Delta m\right)^2+\left(\pdb{\kappa}{A}\cdot\Delta A\right)^2} \\
	&=\num{}
\end{align}
Fitting the model in \autoref{eq:kappa} to the data gives a statistical error of
