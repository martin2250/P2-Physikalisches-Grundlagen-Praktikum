\chapter{Adiabatic Exponent Of Air}
This experiment is used to determine the adiabatic exponent of air, based on \texttt{Rüchardt}'s method.

\section{Theory}
A relatively heavy glass cylinder is placed on top of a gas volume \todo{held captive by a terroristic glass bottle}.
This cylinder is excited to vibrations.
As the frequency of these vibrations tend to be too high to be measured by eye, a combination of a coil, a magnet and a frequency counter is used as an electronic solution.

The periodic expansion/compression of this gas is approximately adiabatic. It holds
\begin{align}
	&pV^{\kappa} &&= \text{const.}& \nonumber \\
	\Leftrightarrow\ &\d p &&= -\kappa\cdot p\frac{\d V}{V}&&|\cdot A \nonumber \\
	\Leftrightarrow\ &\d F &&= -\kappa A^2\cdot\frac{p}{V}\cdot\d x &&\text{(using $A\cdot \d p=\d F$ and $A\cdot\d x=\d V$).} \label{eq:adiabatic_df}
\end{align}\todo{fix shitty left alignment of align}

\autoref{eq:adiabatic_df} is a linear law with an eigenfrequency of
\begin{align*}
	f&=\frac{1}{2\pi}\sqrt{\frac{k}{m}},
\end{align*}
where $k=\kappa A^2\cdot\sfrac{p}{V}$.
This results in
\begin{align}\label{eq:kappa}
	\kappa&=\left(\frac{2\pi f}{A}\right)^2\cdot\frac{mV}{p},
\end{align}
where\par
\begin{tabular}{ll}
	$f-$	&	frequency of oscillation \\
	$m-$	&	cylinder mass (including magnet) \\
	$A-$	&	cross section \\
	$V-$	&	gas volume \\
	$p-$	&	gas pressure (accounting for magnet mass)\\
\end{tabular}\par
denote the characteristic properties of the experiment.
