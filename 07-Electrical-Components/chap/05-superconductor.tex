%!TEX root = ../07-Electrical-Components.tex
\chapter{Superconductivity}
This experiment explores superconductivity.

\section{Theory}
Superconductivity is a quantum mechanical phenomenon, in which the so called \textbf{superconductor}'s electrical resistance abruptly decreases to exactly zero when cooled below their so called \textbf{critical temperature}.
For example an electric current flowing through a superconductor loop can persist infinitely long without a power source whatsoever.
Furthermore, superconductors expel any magnetic fields from them, which is called the \texttt{Meissner-Ochsenfeld}-effect.
Until 1986, physicists believed that superconductivity is not possible to occur above \SI{30}{\kelvin}.
However \texttt{Bednorz} and \texttt{Müller} (nobel prize 1987) proved this statement to be wrong by first finding that a lanthanum-based cuprate perovskite material had its critical temperature at \SI{35}{\kelvin}.
This temperature could further be raised by replcaing the lanthanum by yttrium, which raised the critical temperature to \SI{92}{\kelvin}\cite{PhysRevLett.58.908}, which marked the birth of high temperature superconductors.
A big advantage over ordinary superconductors is that they can be cooled with liquid nitrogen, which is widely available.

\section{Evaluation}
\begin{figure}[tbp]
	\centering
	\includegraphics[width=0.5\textwidth]{./data/plots/superconductor.pdf}
	\caption[Superconductivity: Resistance over temperature]{\text{Resistance over temperature curve} for a high temperature superconductor}
	\label{fig:superconductor}
\end{figure}
\autoref{fig:superconductor} shows the resistance over temperature curve for a non-specified high temperature superconductor.
The critical temperature is approximately \SI{148}{\kelvin}.
The resistance edge is clearly visible at this temperature which confirms our expectations.
