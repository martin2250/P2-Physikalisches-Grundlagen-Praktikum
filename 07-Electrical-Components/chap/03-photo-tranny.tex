%!TEX root = ../07-Electrical-Components.tex
\chapter{Phototranny}
This experiment explores the phototransistor and its behavior in dependence on \todo{well... light}.

\section{Description}
A phototransistor works like a common transistor, with the difference that its base current is controlled by incident light.
Phototransistors are much more sensitive than photodiodes because they also act as amplifiers, which makes a higher resolution possible.
Typical applications are photodetectors, light barriers and optical couplers.
Despite their sensitivity, phototransistors are slow in comparison with photodiodes.

\section{Evaluation}
\autoref{subfig:meas_phototranny} shows the phototransistor measurements for varying lamp luminosities.
The reverse current can easily be determined by the current at the pleateau regions.
Plotting these reverse currents versus their corresponding luminosities yields \autoref{subfig:phototranny_curve}.
It is easy to see that the current saturates with rising luminosity, which confirms our expectations.
For high luminosities, all available electrons on the base-collector path are excited and thus an increase of luminosity does not affect the reverse current anymore.
\begin{figure}[btp]
	\centering
	\begin{subfigure}{0.4\textwidth}
		\centering
		\includegraphics[width=1.05\textwidth]{./data/plots/phototranny.pdf}
		\caption[Measured phototransistor characteristics for variable lamp luminosity]{\textbf{Measured phototransistor characteristics for variable lamp luminosity} Reverse current can be determined by the plateau current}
		\label{subfig:meas_phototranny}
	\end{subfigure}\quad
	\begin{subfigure}{0.4\textwidth}
		\centering
		\includegraphics[width=1.05\textwidth]{./data/plots/phototranny_curve.pdf}
		\caption[Reverse current over luminosity]{\textbf{Reverse current over luminosity} A saturation is visible for high luminosities}
		\label{subfig:phototranny_curve}
	\end{subfigure}
	\caption{Phototransistor behavior in dependence of incident luminosity}
\end{figure}
