%!TEX root = ../07-Electrical-Components.tex
\chapter{Phototranny}
This experiment explores the phototransistor and its behavior in dependence on \todo{well... light}.

\section{Description}
A phototransistor works like a common transistor, with the difference that its base current is controlled by incident light.
Phototransistors are much more sensitive than photodiodes because they also act as amplifiers, which makes a higher resolution possible.
Typical applications are photodetectors, light barriers and optical couplers.
Despite their sensitivity, phototransistors are slow in comparison with photodiodes.

\section{Evaluation}
\autoref{subfig:meas_phototranny} shows the phototransistor measurements for varying lamp illuminances.
The reverse current can easily be determined by the current at the pleateau regions.
Plotting these reverse currents versus their corresponding illuminances yields \autoref{subfig:phototranny_curve}.
It is easy to see that the current saturates with rising illuminance, which confirms our expectations.
For high illuminances, all available electrons on the base-collector path are excited and thus an increase of illuminance does not affect the reverse current anymore.
\begin{figure}[btp]
	\centering
	\begin{subfigure}{0.4\textwidth}
		\centering
		\includegraphics[width=1.05\textwidth]{./data/plots/phototranny.pdf}
		\caption[Measured phototransistor characteristics for variable lamp illuminance]{\textbf{Measured phototransistor characteristics for variable lamp illuminance} Reverse current can be determined by the plateau current}
		\label{subfig:meas_phototranny}
	\end{subfigure}\quad
	\begin{subfigure}{0.4\textwidth}
		\centering
		\includegraphics[width=1.05\textwidth]{./data/plots/phototranny_curve.pdf}
		\caption[Reverse current over illuminance]{\textbf{Reverse current over illuminance} A saturation is visible for high illuminances}
		\label{subfig:phototranny_curve}
	\end{subfigure}
	\caption{Phototransistor behavior in dependence of incident illuminance}
\end{figure}
