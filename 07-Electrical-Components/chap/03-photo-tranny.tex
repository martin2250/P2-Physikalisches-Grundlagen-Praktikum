%!TEX root = ../07-Electrical-Components.tex
\chapter{Phototransistor}
This experiment explores the phototransistor and its characteristics under illumination.

\section{Description}
A phototransistor works like a common transistor, with the difference that its base current is controlled by incident light.
Phototransistors are much more sensitive than photodiodes because they also act as amplifiers, which makes a lower detection threshold possible.
Typical applications are photodetectors, light barriers and optoisolators.
Despite their sensitivity, phototransistors are slow in comparison to photodiodes due to the \texttt{Miller} effect (\cite{miller}) and are not able to detect low light levels any better.

\section{Evaluation}
\autoref{subfig:meas_phototranny} shows the phototransistor's characteristic curve for varying light intensity.
The reverse current can easily be determined by the current at the pleateau regions.
Plotting these reverse currents versus their corresponding illuminances yields \autoref{subfig:phototranny_curve}.
A saturation of the slope is to be expected, as, for high illuminances, all available electrons on the base-collector path are excited and thus an increase of illuminance would not affect the reverse current anymore.
However, this saturation is not to be seen in \autoref{subfig:phototranny_curve}, it rather becomes linear.
Measuring for higher illuminances would increase the significance of the data, as the behavior of the slope could be examined further.

\begin{figure}[btp]
	\centering
	\begin{subfigure}{0.4\textwidth}
		\centering
		\includegraphics[width=1.05\textwidth]{./data/plots/phototranny.pdf}
		\caption[Measured phototransistor characteristics for variable lamp illuminance]{\textbf{Measured phototransistor characteristics for variable lamp illuminance} Reverse current can be determined by the plateau current}
		\label{subfig:meas_phototranny}
	\end{subfigure}\quad
	\begin{subfigure}{0.4\textwidth}
		\centering
		\includegraphics[width=1.05\textwidth]{./data/plots/phototranny_curve.pdf}
		\caption[Reverse current over illuminance]{\textbf{Reverse current over illuminance} A saturation is visible for high illuminances}
		\label{subfig:phototranny_curve}
	\end{subfigure}
	\caption{Phototransistor behavior in dependence of incident illuminance}
\end{figure}
