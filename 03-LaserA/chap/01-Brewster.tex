% !TEX root = ../main.tex
\chapter{Brewster-Angle}
To measure the Brewster-angle discussed in \autoref{sec:brewster} two methods can be used.

\section{By Maximum}
The laser is pointed at a transparent, glass-like material.
If the laser beam hits the surface of the material at Brewster-angle, the p-polarised portion is not reflected and thus full intensity is transmitted.
Like this, a maximum is observed at an angle of \SI{63}{\degree}.
Using \autoref{eq:brewster} and approximating the refractive index of air as $1$, the material's refractive index is determined as
\begin{equation*}
	n_{2,1}\approx\num{1.963}.
\end{equation*}
Due to lack of any information concering the used material, no comparison with a literature value is possible.

\section{By Minimum}
This time the material is positioned so that it reflects the beam with minimal intensity.
The minimum is observed at an angle of \SI{65}{\degree}, which makes a refractive index of
\begin{equation*}
	n_{2,2}\approx\num{2.145}.
\end{equation*}

\section{Evaluation}
Averaging both values, a mean refractive index of $\approx\num{2.054}$ is calculated.
This value does not resemble any common values of materials used for optics.
A quite high chance of errors is to be expected, since the maximum and minimum were both determined by eye.
Errors could have been minimized by using a photo diode or the like.
Furthermore, a new guy was doing the experiment, so it's totally his fault.
