% !TEX root = ../main.tex
\chapter{Diffraction with single slit, bridge, circular disk and opening}

\section{Single-slit}
To determine the gap width of two single slits, which were specified as \SI{0.2}{\milli\meter} and \SI{0.3}{\milli\meter}, the distance between intensity troughs of the diffraction patterns is measured.
Calculations as described in \autoref{sec:meas} were made.
Tables \ref{tab:single_slit_a} and \ref{tab:single_slit_b} show the measured data.
Using the diffraction method,
\begin{itemize}
	\item the gap width of the \SI{0.2}{\milli\meter}-slit is determined as \SI{0.181(3)}{\milli\meter} \newline($\rightarrow$ rel. dev. \num{9.5}\%)
	\item whereas the width of the \SI{0.3}{\milli\meter}-slit is determined as \SI{0.259(7)}{\milli\meter} \newline($\rightarrow$ rel. dev. \num{13.6}\%).
\end{itemize}
That is, the measured values do not remain within error tolerances.
This can be explained by taking into account that the positions of intensity troughs were marked by hand and eye.
Moreover, the given gap widths could be unprecise, no error tolerances were specified.
\begin{table}[b!]
	\centering
	\caption{Single slit gap width, $b=\SI{0.2}{\milli\meter}$}
	\label{tab:single_slit_a}
	%\begin{tabular}{SSS}
	\begin{tabular}{
	S[table-format=-1.2, table-align-exponent = false, table-align-uncertainty = false]
	S[table-format=-1.3(3)e2, table-align-exponent = false, table-align-uncertainty = false]
	S[table-format=-1.3(3)e2, table-align-exponent = false, table-align-uncertainty = false]
	}
		\toprule
		{$D_\text{screen}$ (\si{\meter})}&	{$S$ (\si{\radian\per\order})}&	{$W$ (\si{\meter})}\\
		\midrule
			2.45&   3.61(13)e-3&    1.75( 6)e-4\\
			2.18&   3.48(15)e-3&    1.82( 8)e-4\\
			2.18&   3.47(15)e-3&    1.82( 8)e-4\\
			2.18&   3.40(14)e-3&    1.86( 8)e-4\\
			1.48&   3.52(21)e-3&    1.80(11)e-4\\
		\midrule
			{mean}& 3.50( 7)e-3&    1.81( 3)e-4\\
		\bottomrule
	\end{tabular}
\end{table}

\begin{table}[b!]
	\centering
	\caption{Single slit gap width, $b=\SI{0.3}{\milli\meter}$}
	\label{tab:single_slit_b}
	%\begin{tabular}{SSS}
	\begin{tabular}{
	S[table-format=-1.2, table-align-exponent = false, table-align-uncertainty = false]
	S[table-format=-1.3(3)e2, table-align-exponent = false, table-align-uncertainty = false]
	S[table-format=-1.3(3)e2, table-align-exponent = false, table-align-uncertainty = false]
	}
		\toprule
		{$D_\text{screen}$ (\si{\meter})}&	{$S$ (\si{\radian\per\order})}&	{$W$ (\si{\meter})}\\
		\midrule
		2.45&   2.41(13)e-3&    2.62(14)e-4\\
		2.25&   2.49(14)e-3&    2.54(14)e-4\\
		2.05&   2.47(15)e-3&    2.57(16)e-4\\
		1.85&   2.42(17)e-3&    2.61(18)e-4\\
		1.65&   2.38(19)e-3&    2.66(21)e-4\\
		\midrule
		{mean}& 2.44( 7)e-3&    2.59( 7)e-4\\
		\bottomrule
	\end{tabular}
\end{table}

\section{Bridge}
As elaborated in \autoref{subsec:bridge}, the diffraction patterns of a bridge and a single-slit of same gap width should look alike.
After placing a bridge of width \SI{0.3}{\milli\meter} in the beam path, Babinet's principle can be confirmed, as the minima in the bridge's and the single slit's diffraction pattern coincide.

\section{Circular disk and opening}
\subsection*{Disk}\label{subsec:disk}
First, a circular disk is placed in the beam path.
The observed pattern looks just like one would expect it to look:
A rotational-solid of the bridge's diffraction pattern viz. a bright maximum in the middle, which is called the \textbf{Arago spot} and small rings around it.
The Arago spot is caused by the fact that all diffracted beams around the disk have the same distance from the optical axis.
Therefore, they interfere constructively at the middle of the pattern.

\subsection*{Opening}
After placing a circular opening in the beam path, no specific diffraction pattern could be observed, although Babinet's principle guarantees the pattern to look like the pattern in \autoref{subsec:disk}.
At first, the used diapositive was suspected to be dirty.
This suspection was confirmed by a closer look.
After cleaning the slide with \todo{Isopropanol?} the dirt traces still remained.
We recommend that the lab supervisors should replace the slides with newer ones.

\section{Hair}
