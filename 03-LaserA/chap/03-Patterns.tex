% !TEX root = ../main.tex
\chapter{Repeating Patterns}
\section{Width and Spacing of a Double Slit}

A double slit is placed in the beam path.
The spacing of the two slits is determined by measuring the distance between intensity peaks of the diffraction pattern.
To also determine the width of the individual slits, only a single slit is illuminated and the distance between troughs is measured.
Peaks or troughs are used, as they are more narrow using the respective setup, which makes them easier to measure.
As the distance between two adjacent extremes is the same for peaks and troughs, as well as single and double slits, the same formula \todo{autoref} applies.

\begin{table}[b!]
	\centering
	\caption{Double Slit Width}
	\begin{tabular}{SSS}
		\toprule
		{$D_\text{screen}$ (\si{\meter})}&	{$S$ (\si{\radian\per\order})}&	{$W$ (\si{\micro\meter})}\\
		\midrule
		2.45&	3.23(13)e-3&	1.96( 8)e-4\\
		2.25&	3.18(14)e-3&	1.99( 9)e-4\\
		2.05&	3.16(16)e-3&	2.00(10)e-4\\
		1.85&	3.20(17)e-3&	1.98(11)e-4\\
		1.65&	3.18(20)e-3&	1.99(12)e-4\\
		\midrule
		{mean}&	3.19( 7)e-3&	1.98( 4)e-4\\
		\bottomrule
	\end{tabular}
\end{table}

\begin{table}[b!]
	\centering
	\caption{Double Slit Spacing}
	\begin{tabular}{SSS}
		\toprule
		{$D_\text{screen}$ (\si{\meter})}&	{$S$ (\si{\radian\per\order})}&	{$W$ (\si{\micro\meter})}\\
		\midrule
		2.45&	2.49(13)e-3&	2.54(13)e-4\\
		2.25&	2.51(14)e-3&	2.52(14)e-4\\
		2.05&	2.49(16)e-3&	2.55(16)e-4\\
		1.85&	2.82(17)e-3&	2.25(14)e-4\\
		1.65&	2.49(20)e-3&	2.54(20)e-4\\
		\midrule
		{mean}&	2.55( 7)e-3&	2.49( 7)e-4\\
		\bottomrule
	\end{tabular}
\end{table}
