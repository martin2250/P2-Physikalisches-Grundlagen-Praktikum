% !TEX root = ../main.tex
\chapter{Polarisationsarten}
In diesem Versuchtsteil wird mithilfe eines optischen Systems verschieden polarisiertes Licht erzeugt und die hinter dem abschließenden Analysator auftretende Intensität in Abhängigkeit seiner Stellung gemessen.
Als Detektor wird hierbei ein Fototransistor verwendet.
Bei dem optischen Aufbau ist es von großer Wichtigkeit, alle optischen Elemente auf einer gemeinsamen Achse zu platzieren, was sich am Versuchstag als problematisch erwiesen hat, da die Reiter und optischen Elemente auf der Bank nicht einheitlich waren.\par
Linear polarisiertes Licht kann mithilfe eines Polarisationsfilters erzeugt werden, dabei spielt die Wellenlänge des Lichtes keine Rolle, weshalb polychromatisches Licht verwendet werden kann.
Elliptisch und zirkular polarisiertes Licht kann mithilfe eines anisotropen Mediums (zum Beispiel einem Glimmerplättchen) erzeugt werden, jedoch hängen die Polarisationseigenschaften hier von der Wellenlänge ab.
Daher muss ein Bandpassfilter vorgeschaltet werden, welcher monochromatisches Licht erzeugen kann.

\section{Lineare Polarisation}
\begin{figure}[tb]
	\begin{subfigure}{.4\textwidth}
		\centering
		\includegraphics[height=.8\linewidth]{./data/plots/linear_ohne_filter.pdf}
		\caption[ohne Filter]{Linear polarisiertes Licht, ohne Filter}
		\label{subfig:lin_nofilt}
	\end{subfigure}
	$\quad$
	\begin{subfigure}{.4\textwidth}
		\centering
		\includegraphics[height=.8\linewidth]{./data/plots/linear_mit_filter.pdf}
		\caption[mit Filter]{Linear polarisiertes Licht, mit Filter}
		\label{subfig:lin_withfilt}
	\end{subfigure}
	\caption[Linear polarisiertes Licht]{Messungen des linear polarisierten Lichtes in Abhängigkeit der Stellung des Analysators}
	\label{fig:meas_lin}
\end{figure}

\subsection{Ohne Filter}
In der Theorie sollte ein ideales Polarisationsfilter dazu in der Lage sein, linear polarisiertes Licht ungeachtet seiner Wellenlänge zu filtern.
Wird in einem Polardiagramm die Spannung des Fotodetektors gegen den Winkel des Analysators aufgetragen, so sollte der entstandene Graph einer Acht ähneln, doch dies ist, wie in \autoref{subfig:lin_nofilt} dargestellt, nicht der Fall.
Es ist gut zu erkennen, dass trotz orthogonaler Stellung des Analysators zum Polfilter immer noch Licht zum Detektor durchdringt.
Daraus lässt sich schließen, dass das Polarisationsfilter nicht alle Wellenlängen des Lichtes gleich gut filtern kann.
Läge ein ideales Polfilter vor, wäre ein ähnliches Messergebnis wie in \autoref{subfig:lin_withfilt} zu erwarten.
Außerdem ist eine kleine Neigung von etwa \SI{5}{\degree} des Graphen ersichtlich, was auf eine Abweichung der Ausrichtungsskala des Analysators von der tatsächlichen Stellung schließen lässt.
Trotz der Bemühungen, beide Filter bezüglich ihrer Stellung abzugleichen, kann also eine minimale Fehlstellung nicht vermieden werden.
Dies ist der Tatsache verschuldet, dass der Abgleich per Hand vorgenommen wird, was mitunter nicht perfekt möglich ist.

\subsection{Mit Filter}
Wird nun ein Interferenzfilter vor die Anordnung geschaltet, so ergibt sich der Graph in \autoref{subfig:lin_withfilt}, welcher nah an der zu erwartenden Abhängigkeit liegt.
Das Minimum der Intensität liegt fast bei Null und der Graph beschreibt eine Acht.
Mögliche Fehlerquellen sind ein nicht gänzlich abgedunkelter Raum und die Breite des Passbandes des Filters, diese Quellen konnten weitgehend vernachlässigt werden.
%Da der Fotodetektor zuvor durch Verschließen seines Lichteinlasses auf seine Funktion geprüft wurde, konnte auch dieser Fehler vermieden werden. \todo{'kleine Fehlfunktion, Lichteinlass' sound stupid, please fix}
Aufgrund der guten Ergebnisse werden die nächsten Versuche mit vorgeschaltetem Interferenzfilter durchgeführt.

\section{Elliptische Polarisation}\label{sec:ellip}
\begin{figure}[tb]
	\begin{subfigure}{.4\textwidth}
		\centering
		\includegraphics[height=.8\linewidth]{./data/plots/elliptisch_45-50um.pdf}
		\caption[\SIrange{45}{50}{\micro\meter}]{Elliptisch polarisiertes Licht, \SIrange{45}{50}{\micro\meter}}
		\label{subfig:ellip_45_50}
	\end{subfigure}
	$\quad$
	\begin{subfigure}{.4\textwidth}
		\centering
		\includegraphics[height=.8\linewidth]{./data/plots/elliptisch_60-65um.pdf}
		\caption[\SIrange{60}{65}{\micro\meter}]{Elliptisch polarisiertes Licht, \SIrange{60}{65}{\micro\meter}}
		\label{subfig:ellip_60_65}
	\end{subfigure}
	\caption[Elliptisch polarisiertes Licht]{Messungen des elliptisch polarisierten Lichtes in Abhängigkeit der Stellung des Analysators}
	\label{fig:meas_ellip}
\end{figure}
Zur Erzeugung von elliptisch polarisiertem Licht wird zwischen Polarisator und Analysator ein Glimmerplättchen gebracht.
Es wird für zwei Glimmerplättchen verschiedener Dicke gemessen und erneut ihre Intensität in Abhängigkeit vom Winkel zwischen Polfilter und Analysator aufgenommen.
Zunächst werden dabei vor dem Hereinbringen des Glimmers Analysator und Polfilter um \SI{90}{\degree} zueinander verdreht.
Der Glimmer angebracht und so gedreht, dass am Detektor erneut kein Licht ankommt.
Dann wird der Polarisator um \SI{45}{\degree} gedreht, um gerade zwischen den beiden optischen Achsen zu stehen.
An den Graphen in \autoref{fig:meas_ellip} fällt zunächst auf, dass sie um \SI{45}{\degree} gegenüber den Graphen in \autoref{fig:meas_lin} gedreht sind, was aufgrund der elliptischen Polarisation zu erwarten war.
Außerdem geht die Intensität selbst mit Interferenzfilter nie auf \SI{0}{\volt} zurück, was ebenfalls an dem elliptischen Charakter liegt.

\section{Zirkulare Polarisation}
\begin{figure}
	\centering
	\includegraphics[width=0.7\textwidth]{./data/plots/zirkular.pdf}
	\caption[Zirkular polarisiertes Licht]{Messungen des zirkular polarisierten Lichtes in Abhängigkeit der Stellung des Analysators}
	\label{fig:zircular}
\end{figure}

Es wird mithilfe eines $\frac{\lambda}{4}$-Plättchens zirkular polarisiertes Licht erzeugt.
Dieses wird zwischen Polfilter und Analysator in den Strahlengang gebracht und die gleiche Kalibrierung wie in \autoref{sec:ellip} vorgenommen.
\autoref{fig:zircular} zeigt die Messergebnisse.
Die typischen Charakteristika für zirkular polarisiertes Licht sind gut ausgeprägt:
Nahezu unveränderte Intensität in Abhängigkeit vom Winkel und keine bevorzugte Neigung des Graphen in eine Richtung.
Die minimale Abweichung von der kreisrunden Form des Graphen lässt sich durch die Breite des Passbandes des Filters erklären.
