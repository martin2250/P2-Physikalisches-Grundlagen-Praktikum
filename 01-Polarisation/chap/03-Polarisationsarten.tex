% !TEX root = ../main.tex
\chapter{Polarisationsarten}
In diesem Versuchtsteil wird mithilfe eines optischen Systems verschieden polarisiertes Licht erzeugt und die hinter dem abschließenden Analysator auftretende Intensität in Abhängigkeit seiner Stellung gemessen.
Als Detektor wird hierbei ein Fototransistor verwendet.
Bei dem optischen Aufbau ist es von großer Wichtigkeit, alle optischen Elemente auf einer gemeinsamen Achse zu platzieren, was sich am Versuchstag als problematisch erwiesen hat, da die Reiter und optischen Elemente auf der Bank nicht einheitlich waren.\par
Linear polarisiertes Licht kann mithilfe eines Polarisationsfilters erzeugt werden, dabei spielt die Wellenlänge des Lichtes keine Rolle, weshalb polychromatisches Licht verwendet werden kann.
Elliptisch und zirkular polarisiertes Licht kann mithilfe eines anisotropen Mediums (zum Beispiel einem Glimmerplättchen) erzeugt werden, jedoch hängt hier die Doppelbrechung von der Wellenlänge ab.
Daher muss ein Interferenzfilter vorgeschaltet werden, welcher monochromatisches Licht erzeugen kann.

\section{Lineare Polarisation}
\begin{figure}[tb]
	\begin{subfigure}{.4\textwidth}
		\centering
		\includegraphics[height=.8\linewidth]{./data/plots/linear_ohne_filter.pdf}
		\caption[ohne Filter]{Linear polarisiertes Licht, ohne Filter}
	\end{subfigure}
	$\quad$
	\begin{subfigure}{.4\textwidth}
		\centering
		\includegraphics[height=.8\linewidth]{./data/plots/linear_mit_filter.pdf}
		\caption[mit Filter]{Linear polarisiertes Licht, mit Filter}
	\end{subfigure}
	\caption[Linear polarisiertes Licht]{Messungen des linear polarisierten Lichtes in Abhängigkeit der Stellung des Analysators}
	\label{fig:meas_lin}
\end{figure}
