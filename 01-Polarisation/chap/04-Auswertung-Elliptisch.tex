% !TEX root = ../main.tex
\chapter{Bestimmung der Brechzahldiffernz der Phasenplättchen}

Es soll die Brechzahldiffernz zwischen den extraordinären Strahlen der gegebenen Phasenplättchen aus den beiden Messungen an elliptisch polarisiertem Licht bestimmt werden.
\autoref{eq:phasediff} gibt den Zusammenhang zwischen der Phasendifferenz, der Dicke der Plättchen und der Brechzahldiffernz an.
Die Dicke der Plättchen ist bekannt, es bleibt die Phasendifferenz zu ermitteln.

Im Allgemeinen ist diese schwer zu bestimmen, durch die Einschränkung gleicher Amplituden des ordinären und extraordinären wird dies vereinfacht und es gilt der Zusammenhang
\begin{equation}
	\Delta \varphi = 2 \cdot \arctan\sqrt{\frac{T}{L}},
\end{equation}
mit der Tallienweite $T$ und der Länge $L$ der Figur.
Dieser Zusammenhang lässt sich der Vorbereitungsmappe entnehmen.

Die Messwerte sind in \autoref{tab:tabtab} aufgelistet.
Mit \autoref{eq:phasediff} kann schließlich die Brechzahldiffernz bestimmt werden.
Da die Dicke der Glimmerplättchen nur bereichsweise angegeben ist, werden Brechzahlbereiche berechnet.

\begin{table}
	\centering
	\caption{Zur Brechzahlbestimmung}
	\label{tab:tabtab}
	\begin{tabular}{ccccc}
	\toprule
	&	{$L$ (\si{\milli\volt})}&	{$T$ (\si{\milli\volt})}&	{$\Delta \varphi$}&	{$\Delta n$}\\
	\midrule
	\SIrange{45}{50}{\micro\meter}& 268&	\num{13.4}&	\SI{5.7}{\degree}& \num{11.5e-3} bis \num{12.8e-3}\\
	\SIrange{60}{65}{\micro\meter}& 300&	22&	\SI{8.4}{\degree}& \num{13.1e-3} bis \num{14.1e-3}\\
	\bottomrule
	\end{tabular}
\end{table}

In \autoref{tab:tabtab} ist ersichtlich, dass die Brechzahldifferenzen nicht nur in der gleichen Größenordnung liegen, sondern recht nah beieinander sind.
Dieses Ergebnis entspricht den Erwartungen, da bei beiden Messreihen das gleiche Material verwendet wird.
Ein Vergleich mit den Literaturwerten\footnote{Quelle:\hyperlink{https://de.wikipedia.org/wiki/Doppelbrechung}{de.wikipedia.org/wiki/Doppelbrechung}} für Muskovit (Glimmer) bestätigt zumindest die Größenordnung unserer Ergebnisse:
$\Delta n_{\text{Lit,}\beta\gamma}=\num{5e-3}$, damit weichen die gemessenen Werte um etwa \num{150}\% vom Literaturwert ab.
Mögliche Fehlerquellen sind ein nicht gänzlich abgedunkelter Raum und die Breite des Passbandes des Filters.
