% !TEX root = ../main.tex
\chapter{Bestimmung der Brechzahldiffernz der Phasenplättchen}

Es soll die Brechzahldiffernz zwischen den extraordinären Strahlen der gegebenen Phasenplättchen aus den beiden Messungen an ellitisch polarisiertem Licht bestimmt werden.
\autoref{eq:phasediff} gibt den Zusammenhang zwischen der Phasendifferenz, der Dicke der Plättchen und der Brechzahldiffernz an.
Die Dicke der Plättchen ist bekannt, es bleibt die Phasendifferenz zu ermitteln.

Im Allgemeinen ist diese schwer zu bestimmen, durch die Einschränkung gleicher Amplituden des ordinären und extraordinären wird dies vereinfacht, und es gilt der Zusammenhang
\begin{equation}
	\Delta \varphi = 2 \cdot \arctan\sqrt{\frac{T}{L}},
\end{equation}
mit der Tallienweite $T$ und der Länge $L$ der Figur.
Dieser Zusammenhang lässt dich der Vorbereitungsmappe entnehmen.

Die Messwerte sind in \autoref{tab:tabtab} aufgelistet.
Mit \autoref{eq:phasediff} kann schließlich die Brechzahldiffernz bestimmt werden.

%Dn = Dphi/(k*d) = Dphi * lambda/(2pi*d)

\begin{table}
	\centering
	\caption{Zur Brechzahlbestimmung}
	\label{tab:tabtab}
	\begin{tabular}{ccccc}
	\toprule
	&	{$L$}&	{$T$}&	{$\Delta \varphi$}&	{$\Delta \n$}\\
	\midrule
	\SIrange{45}{50}{\micro\meter}& 268&	\num{13.4}&	\SI{5.7}{\degree}& \num{0}\\
	\SIrange{60}{65}{\micro\meter}& 300&	22&	\SI{8.4}{\degree}& \num{0}\\
	\bottomrule
	\end{tabular}
\end{table}