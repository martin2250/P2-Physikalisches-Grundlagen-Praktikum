% !TEX root = ../main.tex
\chapter{Grundlagen}

In diesem Praktikum werden die verschiedenen Polarisationsarten von Licht untersucht.

\section{Polarisation}

Sichtbares Licht ist eine elektromagnetische Welle und ist als solche im Vakuum transversal.
Das heißt, das elektrische und magnetische Feld stehen beide sekrecht zur Ausbreitungsrichtung.
Im Alltag sind oft Überlagerungen von Wellen mit vielen unterschiedlichen Polarisationsrichtungen (Richtung des $\vec{E}$-Feld Vektors) zu finden, dabei spricht man von unpolarisiertem Licht.

Mithilfe von optischen Apparaturen kann allerdings auch Licht hegestellt werden, welches eine regelmäßige Polarisation aufweist.
Dabei werden folgende Arten unterschieden:

\begin{itemize}
	\item \textbf{Linear polarisiertes Licht} weist eine konstante Richtung der Polarisationsachse auf.
	Moderne Polarisationsfilter für lineare Polarisation weisen eine von der Polarisationsrichtung abhängige optische Dämpfung auf, diese Eigenschaft wird Dichroismus genannt.
	\item \textbf{Zirkular polarisiertes Licht} ist eine Überlagerung von zwei linear polarisierten Lichtwellen gleichen Betrages mit senkrecht zueinander stehenden Polarisationsachsen, welche einen Phasenunterschied von \SI{90}{\degree} aufweisen.
	Dies führt dazu, dass der Betrag des elektrischen Feldes konstant bleibt, die Richtung allerdings in der Ebene senkrecht zur Ausbreitungsrichtung rotiert.
	Hierbei wird zwischen links- und rechtsdrehend unterschieden.
	Zirkular polarisiertes Licht kann wie später beschrieben mithilfe eines $\lambda / 4$-Plättchens aus linear polarisiertem Licht oder ähnlich wie beim linearen Polarisationsfilter mit chiralen Materialien durch gezielte Absorption aus unpolarisiertem Licht gewonnen werden.
	\item \textbf{Elliptisch polarisiertes Licht} ist ebenfalls eine Überlagerung von zwei linear polarisierten Lichtwellen, allerdings können die Beträge und die Phasendifferenz beliebige Werte annehmen, weshalb der $\vec{E}$-Feld Vektor eine elliptische Kurve in der senkrechten Ebene ausführt.
	Lineare und zirkulare Polarisation können als Grenzfall der elliptischen Polarisation angesehen werden.
\end{itemize}

\section{Doppelbrechung}

Doppelbrechung ist ein Effekt, der an optisch anisotropen Materialen auftritt.
Bei isotropen Medien hat der Dieletrizitätstensor die Form $\underline{\underline{\epsilon}} = \epsilon \; \mathbb{I}$, daher ist die Dielektrizitätskonstante unabhängig von der Richtung des angelegten Feldes.
Doppelbrechende Materiale haben abhängig von der Richtung des elektrischen Feldes bzw. der Polarisation eine unterschiedliche Dielektrizitätskonstante und damit auch eine unterschiedliche Brechzahl.

Um die Richtungsabhängigkeit angeben zu können, wurde die optische Achse eingeführt.
Licht, welches senkrecht zur optischen Achse polarisiert ist, wird ordentlicher Strahl, senkrecht dazu polarisiertes Licht wird außerordentlicher Strahl genannt.
Die dazugehörigen Brechungsindizes werden $n_\text{e}$ ('extraordinary') und $n_\text{a}$ genannt.

Wird Licht mit einer Polarisationsachse zwischen der optischen Achse und der senkrechten durch ein doppelbrechendes Medium geschickt, so breiten sich die beiden Anteile mit unterschiedlichen Geschwindigkeiten aus, es entsteht eine Phasendifferenz zwischen ordinärem und extraordinärem Strahl.
Somit kann aus linear polarisiertem Licht elliptisch polarisiertes Licht erzeugt werden.
Dabei gilt folgender Zusammenhang für die Phasendifferenz $\Delta \varphi$:
\begin{equation} \label{eq:phasediff}
	\Delta \varphi = k_0 \cdot d \cdot \left(n_\text{e} - n_\text{o}\right),
\end{equation}
mit der Dicke $d$ des Materials und der Wellenzahl $k_0 = \frac{2 \uppi}{\lambda_0}$.
Mit der richtigen Wahl der Dicke des Mediums und Polarisationsrichtung des einfallenden Lichts kann auch zirkulare Polarisation erreicht werden.
Dazu muss die Polarisationsachse des einfallende Lichts einen Winkel von \SI{45}{\degree} mit der optischen Achse einschließen und weiter eine Phasendifferenz von $\Delta \varphi = \frac{\uppi}{2}$ erzeugt werden.
