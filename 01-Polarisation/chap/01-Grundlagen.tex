% !TEX root = ../main.tex
\chapter{Grundlagen}

In diesem Praktikum werden die verschiedenen Polarisationsarten von Licht untersucht.

\section{Polarisation}

Sichtbares Licht ist eine elektromagnetische Welle und ist als solche im Vakuum transversal.
Das heißt das elektrische und magnetische Feld stehen beide sekrecht zur Ausbreitungsrichtung.
Im Alltag sind oft Überlagerungen von Wellen mit vielen unterschiedlichen Polarisationsrichtungen (Richtung des $\vec{E}$-Feld Vektors) zu finden, dabei spricht man von unpolarisiertem Licht.

Mithilfe von optischen Apparaturen kann allerdings auch Licht hegestellt werden, welches eine regelmäßige Polarisation aufweist.
Dabei werden folgende Arten unterschieden:

\begin{itemize}
	\item \textbf{Linear polarisiertes Licht} weißt eine konstante Richtung der Polarisationsachse auf.
	Moderne Polarisationsfilter für lineare Polarisation weisen eine von der Polarisationsrichtung abhängige optischen Dämpfung auf, diese Eigenschaft wird Dichroismus genannt.
	\item \textbf{Zirkular polarisiertes Licht} ist eine Überlagerung von zwei linear polarisierten Lichtwellen gleichen Betrages mit senkrecht zueinander stehenden Polarisationsachsen, welche einen Phasenunterschied von \SI{90}{\degree} aufweisen.
	Dies führt dazu, dass der Betrag des elektrischen Feldes konstant bleibt, die Richtung allerdings in der Ebene senkrecht zur Ausbreitungsrichtung rotiert.
	Hierbei wird zwischen links- und rechtsdrehend unterschieden.
	Zirkular polarisiertes Licht kann, wie später beschrieben, mithilfe eines $\lambda / 4$-Plättchens aus linear polarisiertem Licht, oder ähnlich wie beim linearen Polarisationsfilter mit chiralen Materialen durch gezielte Absorption aus unpolarisiertem Licht gewonnen werden.
	\item \textbf{Elliptisch polarisiertes Licht} ist ebenfalls eine Überlagerung von zwei linear polarisierten Lichtwellen, allerdings können die Beträge und Phasendifferenz beliebige Werte annehmen, weshalb der $\vec{E}$-Feld Vektor eine elliptische Kurve in der senkrechten Ebene ausführt.
	Lineare und zirkulare Polarisation können als Grenzfall der elliptischen Polarisation angesehen werden.
\end{itemize}