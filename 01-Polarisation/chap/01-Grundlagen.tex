% !TEX root = ../main.tex
\chapter{Grundlagen}

In diesem Praktikum werden die verschiedenen Polarisationsarten von Licht untersucht.

\section{Polarisation}

Sichtbares Licht ist eine elektromagnetische Welle und ist als solche im Vakuum transversal.
Das heißt das elektrische und magnetische Feld stehen beide sekrecht zur Ausbreitungsrichtung.
Im Alltag sind oft Überlagerungen von Wellen mit vielen unterschiedlichen Polarisationsrichtungen (Richtung des $\vec{E}$-Feld Vektors) zu finden, dabei spricht man von unpolarisiertem Licht.

Mithilfe von optischen Apparaturen kann allerdings auch Licht hegestellt werden, welches eine regelmäßige Polarisation aufweist.
Dabei werden folgende Arten unterschieden:

\begin{itemize}
	\item \textbf{Linear polarisiertes Licht,}
	\item \textbf{zirkulare Polarisation und}
	\item \textbf{elliptische Polarisation.}
\end{itemize}
