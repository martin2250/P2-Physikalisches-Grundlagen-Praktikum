%!TEX root = ../09-Photoelectric-Effect.tex
\chapter{Hallwachs Effect}

One of the early experiments demonstrating the photoelectric effect, first carried out by \texttt{Wilhelm Hallwachs}, is repeated.

A zinc plate is sanded to remove contaminations and surface oxide layers.
It is connected to an electrometer and charged up negatively to \SI{2}{\kV} with respect to earth ground.

In (artificial) ambient light the electrometer shows no visible change in potential.
The plate is then illuminated with a mercury vapor lamp, which causes the electrometer's deflection to decay quickly.

The demonstrated \texttt{Hallwachs effect} is a special form of the photoelectric effect, where the target has an excess of electrons.
Confirming the quantum nature of photons, only photons with high frequencies (and therefore high energies) can free electrons from the target.
Most artificial light sources like halogen bulbs and flourescent tubes only emit photons in the visible and infrared spectrum, which have less energy than zinc's workfunction.
The mercury vapor lamps emit a discrete spectrum with a minimum wavelength of \SI{184.45}{\nm}, which is sufficient to free electrons from the plate.

An additional positively charged electrode can be used to collect the freed electrons, preventing them from falling back onto the plate.
The experiment shows that the deflection falls more quickly, confirming the expectation.
