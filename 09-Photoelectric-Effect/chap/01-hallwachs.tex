%!TEX root = ../09-Photoelectric-Effect.tex
\chapter{Hallwachs Effect}

One of the early experiments demonstrating the photoelectric effect, first carried out by \texttt{Wilhelm Hallwachs}, is repeated.

A zinc plate is sanded to remove contaminations and surface oxide layers.
It is connected to an electrometer and charged up negatively with \SI{2}{\kV} relative to earth ground.

In (artificial) ambient light the electrometer shows no visible change in potential.
The plate is then illuminated with a mercury vapor lamp.
The electrometer's deflection quickly decays.\todo{'the' much?}

\todo{Erklären Sie die beobachteten Effekte}

An additional positively charged electrode can be used to collect the freed electrons and speed up the discharge process.\todo{IMHO the process of 'collecting' the electrons does not speed up discharge, the addition of an attractive potential does. It's just a matter of wording really to mean the same thing.}
