% !TEX root = ../main.tex
\chapter{Absorption of Alpha Particles}
\begin{figure}[ht!]
	\centering
	\includegraphics[width=0.7\textwidth]{./data/plots/2.pdf}
	\caption[Alpha radiation over distance]{Alpha radiation over distance}
	\label{fig:alpha_distance}
\end{figure}

This experiment looks into the distance behavior of alpha radiation emitted by the radioactive isotope Am-241.

The alpha-source Am-241 is placed in front of a Geiger-Müller tube and count rates are recorded for multiple distances in \SI{1}{\milli\meter} intervals.
A few interfering effects have to be considered before the measurements:
\begin{itemize}
	\item \textbf{Background radiation:} The previously measured background radiation has to be substracted from the actual measurement.
	\item \textbf{Changing solid angle:} The solid angle covered by the tube as seen from the radiation source is inversely proportional to the distance squared, so the measured count rates have to be corrected with a factor of $\frac{d^2}{4r^2}$, where $r$ and $d$ denote the radius of the tube and the distance between tube and source respectively.
	\item \textbf{Parasitic gamma radiation:} Am-241 also emits some gamma radiation. The percentage of emitted alphas is $\num{98}\%$\footnote{Source: www.nucleide.org/DDEP\_WG/NUCLIDES/AM-241\_tables.pdf}, hence $R_\text{corr}=0.98\cdot R_\text{meas}$.
\end{itemize}

Unlike gammas, alphas can be completely blocked off after a certain distance due to their charge.
\autoref{fig:alpha_distance} shows the results of the measurements after aforementioned corrections.
A small anomaly can be seen at $d_\text{max}=\SI{0.65}{\inch}$ which can be interpreted as the maximal distance the alphas can travel before ionizing completely.
This also confirms our expectations, as alpha particles interact strongly with matter and thus do not travel for very long distances.
The measured low-level radiation beyond $d_\text{max}$ most likely originates from extraordinarily energized alpha particles or secondary gamma radiation, which could not be accounted for through a statistical estimation.

The quality of the experiment could have benefitted by condcucting a reference experiment with a sheet of paper between tube and source, preventing the alpha radiation from reaching the tube to get a baseline reading of the emitted gamma radiation.
