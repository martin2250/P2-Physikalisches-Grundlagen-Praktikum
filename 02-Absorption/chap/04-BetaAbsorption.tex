% !TEX root = ../main.tex
\chapter{Absorption of Beta Radiation}

\begin{figure}[ht!]
	\centering
	\includegraphics[width=0.7\textwidth]{./data/plots/3.pdf}
	\caption{Beta radiation over Absorber Thickness}
	\label{fig:beta_absorption}
\end{figure}

The absorption of beta radiation in aluminum is analyzed by placing a Sr-90 beta source at a fixed distance in front of the tube and placing aluminum sheets of varying thickness between them.

\autoref{fig:beta_absorption} shows the data, corrected for background level and tube dead time.
As hinted by the \todo{vorbereitung}, the attenuation is modelled as a sum of two exponential decays
\begin{equation*}
	N(d) = A_1 e^{-b_1 \cdot d} + A_2 e^{-b_2 \cdot d},
\end{equation*}
resulting in the parameters
\begin{alignat*}{2}
	A_1 &= \SI{31.3}{\cps} \qquad b_1 &= \SI{233.6e-3}{thou^{-1}}\\
	A_2 &= \SI{37.2}{\cps} \qquad b_2 &= \SI{31.15e-3}{thou^{-1}}
\end{alignat*}
