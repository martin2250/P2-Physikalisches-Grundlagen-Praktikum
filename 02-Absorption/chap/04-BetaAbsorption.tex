% !TEX root = ../main.tex
\chapter{Absorption of Beta Radiation}

\begin{figure}[ht!]
	\centering
	\includegraphics[width=0.7\textwidth]{./data/plots/3.pdf}
	\caption{Beta radiation over Absorber Thickness}
	\label{fig:beta_absorption}
\end{figure}

The absorption of beta radiation in aluminium is analyzed by placing a Sr-90 beta source at a fixed distance in front of the tube and placing aluminium sheets of varying thickness between them.

Sr-90 produces a daughter isotope, Y-90, in it's decay. Y-90 is also unstable and decays with a half life time of $\tau_\text{Y-90} = \SI{64}{\hour}$.
As the provided samples are older than the half life time of Sr-90 ($\tau_\text{Sr-90} = \SI{28.2}{a}$), which is also four orders of magnitude greater than that of Y-90, the amount of Y-90 should have reached a nearly stable value, so the activity rates of Y-90 and Sr-90 should be roughly identical.

The literature values for the kinetic energy of the electrons released in the decay are
\begin{equation*}
	E_\text{Sr-90} = \SI{540}{\keV} \qquad \text{and} \qquad E_\text{Y-90} = \SI{2.286}{\MeV}.
\end{equation*}

\autoref{fig:beta_absorption} shows the data, corrected for background level, tube dead time and the additional attenuation from the tube window, which is roughly equal to \SI{12}{\micro\meter} of aluminum.
The last two data points are not used for fitting the model, since the rate only follows the exponential relationship up to \SI{\sim50}{\percent} absorption.
As hinted by the instructions, the attenuation is modelled as a sum of two exponential decays
\begin{equation*}
	N(d) = A_1 e^{-b_1 \cdot d} + A_2 e^{-b_2 \cdot d},
\end{equation*}
resulting in the parameters
\begin{alignat*}{2}
	A_1 &= \SI{32.4}{\cps} \qquad b_1 &= \SI{8.789}{\milli\meter^{-1}}\\
	\text{and } A_2 &= \SI{36.0}{\cps} \qquad b_2 &= \SI{1.172}{\milli\meter^{-1}},
\end{alignat*}
where $A_i$ is the activity of the element and $b_i$ is the coefficient of absorption for the respective energy.
The similar base activities $A_1$ and $A_2$ confirm that the amount of Y-90 should be nearly stable.

The first set of parameters can be identified with the radiation emitted by the decay of Sr-90 and the second set of parameters describes the more energetic radiation emitted by it's daugher isotope Y-90.

The mass coefficient of absorption $\mu_i$ is the ratio of $b_i$ and the density of aluminum $\rho_\text{Al} = \SI{2.71}{\gram\per\cubic\centi\meter}$, the fit parameters yield
\begin{equation*}
	\mu_\text{Sr-90} = \SI{3.24}{\meter\squared\per\kilo\gram} \qquad \text{and} \qquad \mu_\text{Y-90} = \SI{0.43}{\meter\squared\per\kilo\gram}.
\end{equation*}
The empirical formula\footnote{Gleason} for calculating the mass coefficient of absorption $\mu$ (in \si{\centi\meter\squared\per\gram}) for beta radiation with kinetic energy $E$ (in \si{\MeV}) is $\mu = 17 E^{-1.43}$.
Using the provided literature values, this formula yields
\begin{equation*}
	\mu_\text{Sr-90, lit} = \SI{4.1}{\meter\squared\per\kilo\gram} \qquad \text{and} \qquad \mu_\text{Y-90, lit} = \SI{0.52}{\meter\squared\per\kilo\gram},
\end{equation*}
which differ from the experimentally determined vaules by \SI{27}{\percent} and \SI{21}{\percent} respectively.

The relatively high deviation is likely due to the difficulty of fitting two very similar functions to the data set.

The maximum range of the particles can only be estimated.
The range of Sr-90 ends at the slight kink at $R_\text{Sr-90} \approx \SI{250}{\um}$, the range of Y-90 likely ends close beyond \SI{5}{\mm}, so $R_\text{Y-90} \approx \SI{5}{\mm}$ should be a good enough estimate.

The maximum particle energy $W$ (in \si{\MeV}) can be approximated\footnote{Flammersfeld} with $W = 1.92\left(R^2\rho^2 + 0.22 \cdot R\rho\right)^\frac{1}{2}$, with the range of the radiation $R$ (in \si{\cm}) and the density of the absorber $\rho$ (in \si{\gram\per\cm\cubed}).
Using the estimated ranges, this yields
\begin{equation*}
	W_\text{Sr-90} = \SI{0.27}{\MeV} \qquad \text{and} \qquad W_\text{Y-90} = \SI{2.8}{\MeV},
\end{equation*}
which present a \SI{48}{\percent} and \SI{22}{\percent} error.
Considering the wild guesswork used to find the range of the particles, these results are satisfactory.
