% !TEX root = ../main.tex
\addchap{Introduction}

The following experiments will explore the capabilities of various materials to absorb different types of ionizing radiation.

\section{Types of ionizing radiation}
\begin{itemize}
	\item \textbf{$\upalpha$-radiation} is a form of particle radiation consisting of helium-4 nuclei.
	Due to it's size and strong +2 charge, it interacts readily with other matter and has a low penetration depth.

	The fundamental processes responsible for alpha decay are a balance of nuclear and electromagnetic force.
	While the Coulomb force is repelling the alpha particle from the rest of the nucleus, it is restrained from the nuclear force.
	Altough in classical mechanics the alpha particle doesn't have enough energy to escape from the potential well, the quantum tunneling effect allows the alpha to escape and after some time it does.
	\item \textbf{$\upbeta$-radiation}, more specifically $\upbeta^{-}$-radiation, consists of high-energy electrons.
	Often a product of $\upbeta^{-}$ decay, it is created when a neutron decays into a proton, an electron and an electron neutrino.
	Like $\upalpha$-radiation, the particles carry a charge, though the much smaller cross section gives them an increased penetration depth compared to $\upalpha$ particles.
	\item \textbf{$\upgamma$-radiation} is electromagnetic radiation with energies upwards of \SI{100}{\kilo\electronvolt}.
	Unlike $\upalpha$ and $\upbeta$ particles, photons do not carry a charge and thus interact most weakly with other matter.
\end{itemize}

\section{Absorption characteristics of radiation types}
There is a fundamental in the way charged and neutral particles interact with regular matter.

Charged particles like electrons give off their energy through many interactions by ionizing other atoms, elastic collisions or Coulomb scattering.
This leads to a linear relationship between particle energy and distance travelled and notably dictates that charged particles can be shielded entirely.

Photons, on the other hand, tend to give off their energy in single bursts, by the photoelectric effect and pair production, or also by compton scattering.
This leads to an exponential decrease in radiation dose with thicker shields and implies that $\upgamma$-radiation can never be entirely blocked off.

All interactions between radiation and matter get stronger with increased atomic mass of the absorbing material.
This makes heavy elements, such as lead, very popular for radiation shields.

\section{Measurement of radiation}\label{sec:geiger_shit}
The most common method of detecting ionizing particles is to exploit their ionizing properties to create ion-electron pairs.
This is used in Geiger–Müller tubes.
In a sufficiently strong electric field, the pairs will seperate and accelerate towards the anode/cathode.
If the electrons get enough energy, they can ionize secondary atoms.
The resulting avalanche effect amplifies the released charge by a factor of up to $10^8$.
This large charge can be measured as a current spike, which makes it possible to count individual particles.

To prevent a runaway gas discharge induced by the positive feedback from damaging the equipment or blocking out further particles, the discharge is quenched.
Still, the discharge takes a finite time ('dead time') in which the tube can not detect new particles.

\section{Geiger plateau and operating voltage}
There are three major operating modes for a Geiger-Mueller tube:
\begin{itemize}
	\item In the \textbf{ion chamber region}, the electric field is strong enough to separate the electron-ion pairs, but the electrons never reach high enough energies to free secondary electrons.
	\item The \textbf{proportional region}, electrons close to the anode have enough energy to start electron avalanches. The current output is proportional to the energy of radiation.
	\item The \textbf{Geiger plateau} is the operating voltage range where every electron can ionize further gas atoms.
	The resulting gas discharge propagates through the entire tube and only stops when the seperated electrons and ions cancel out the external field sufficiently.
	In this operating mode, the tube is most sensitive and can detect individual paricles.
	Due to the long discharge and large ion migration, this operating mode also has the longest dead time.
\end{itemize}
