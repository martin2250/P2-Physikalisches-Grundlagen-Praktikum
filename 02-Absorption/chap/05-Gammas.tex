% !TEX root = ../main.tex
\chapter{Absorption of Gamma Radiation}
This experiment deals with absorption of gamma radiation emitted by radioisotopes Co-60 and Cs-137.

\section{Absorption in Lead (Co-60)}\label{sec:co}
\begin{figure}[ht!]
	\centering
	\includegraphics[width=0.7\textwidth]{./data/plots/4_1.pdf}
	\caption[Gamma radiation over Pb thickness]{Gamma radiation over Pb thickness}
	\label{fig:gamma_pb_co}
\end{figure}
To characterize it's ability to absorb gamma radiation, a block of lead is placed between a Co-60 source and the Geiger-Müller tube, graudually increasing the block's thickness.
Again, the background radiation level is subtracted from the number of measured events.
Although dead time of the tube should not matter as it is operated within the plateau region, according corrections are made.

\autoref{fig:gamma_pb_co} shows the measured data and confirms the exponential attenuation with thickness
\begin{equation*}
	N = a \cdot e^{-b \cdot d}.
\end{equation*}

Fit parameters are
\begin{gather*}
	a_\text{Co60}=\num{3.76}cps\qquad b_\text{Co60}=\SI{0.05}{\milli\meter}^{-1}.
\end{gather*}

$b_\text{Co60}$ can be identified with the coefficient of absorption, which can be used to calculate the \todo{mass-coefficient of absorption?}
\begin{equation*}
	\alpha_\text{Co60}=\frac{b_\text{Co60}}{\rho_\text{Pb}}=\SI{4.41e-3}{\meter\squared\per\kilogram},
\end{equation*}
where $\rho_\text{Pb}=\SI{11.342}{\gram\per\cubic\centi\meter}$.

Using the determined fit-parameter $b_\text{Co60}$ we can also calculate the half-value layer
\begin{equation*}
	d_\text{hvl,Co60}=\frac{\log{2}}{b_\text{Co60}}=\SI{12.8}{\milli\meter}.
\end{equation*}

The literature value\footnote{\url{https://www.nde-ed.org/EducationResources/CommunityCollege/Radiography/Physics/HalfValueLayer.htm}} for gamma radiation from a Co-60 source is $d_\text{hvl,Co60}^{*}=\SI{0.49}{\inch}$ .
This makes a relative deviation of \SI{2.4}{\percent}, which is a satisfactory result.

\section{Absorption in Lead (Cs-137)}\label{sec:cs}
\begin{figure}[ht!]
	\centering
	\includegraphics[width=0.7\textwidth]{./data/plots/4_2.pdf}
	\caption[Gamma radiation over Pb thickness]{Gamma radiation over Pb thickness}
	\label{fig:gamma_pb_cs}
\end{figure}

The experiment in \autoref{sec:co} is repeated with the same corrections and requirements, this time using the lower energy radioisotope Cs-137.
\autoref{fig:gamma_pb_cs} shows the measured data and, once again, confirms exponential attenuation.

Fit parameters are
\begin{gather*}
	a_\text{Cs137}=\num{10.74}cps\qquad b_\text{Cs137}=\SI{0.102}{\milli\meter}^{-1},
\end{gather*}
which gives
\begin{equation*}
	\alpha_\text{Cs137}=\frac{b_\text{Cs137}}{\rho_\text{Pb}}=\SI{8.99e-3}{\meter\squared\per\kilogram},
\end{equation*}
where $\rho_\text{Pb}=\SI{11.342}{\gram\per\cubic\centi\meter}$.

This results in
\begin{equation*}
	d_\text{hvl,Cs137}=\frac{\log{2}}{b_\text{Cs137}}=\SI{6.8}{\milli\meter}.
\end{equation*}

The literature value\footnote{\url{http://researchcompliance.uc.edu/Libraries/Isotopes/Cs-137.sflb.ashx}} for gamma radiation from a Cs-137 source is $d_\text{hvl,Cs137}^{*}=\SI{0.3}{\inch}$ .
This makes a relative deviation of \SI{15}{\percent}, which is acceptable.

\section{Discussion of energy dependence}
The experiments conducted in \autoref{sec:co} and \autoref{sec:cs} show that the half-value layer decreases with decreasing energy of gamma rays.
This can be explained by dropping cross section with rising particle energy and increasingly more interactions with low energy attenuation inside the material (e.g. pair formation).

\section{Absorption in various materials}
