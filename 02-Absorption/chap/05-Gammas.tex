% !TEX root = ../main.tex
\chapter{Absorption of Gamma Rdiation}
This experiment deals with absorption of gamma radiation emitted by radioisotopes Co-60 and Cs-137.

\section{Absorption in Lead}
\begin{figure}[ht!]
	\centering
	\includegraphics[width=0.7\textwidth]{./data/plots/4_1.pdf}
	\caption[Gamma radiation over Pb thickness]{Gamma radiation over Pb thickness}
	\label{fig:gamma_pb}
\end{figure}
To characterize it's ability to absorb gamma radiation, a block of lead is placed between a Co-60 source and the Geiger-Müller tube, graudually increasing the block's thickness.
Again, the background radiation level is subtracted from the number of measured events.
Dead time of the tube does not have to be considered, as the tube is operated within the pleateau region.\todo{We'll see that once I've done 1.4.}

\autoref{fig:gamma_pb} shows the measured data and confirms the exponential attenuation with thickness
\begin{equation*}
	N = a \cdot e^{-b \cdot d}.
\end{equation*}

Fit parameters are
\begin{gather*}
	a=\num{3.76}cps\qquad b=\SI{0.05}{\milli\meter}^{-1}.
\end{gather*}

$b$ can be identified with the coefficient of absorption, which can be used to calculate the \todo{mass-coefficient of absorption?}
\begin{equation*}
	\alpha=\frac{b}{\rho_\text{Pb}}=\SI{4.41e-3}{\meter\squared\per\kilogram},
\end{equation*}
where $\rho_\text{Pb}=\SI{11.342}{\gram\per\cubic\centi\meter}$.

Using the determined fit-parameter $b$ we can also calculate the half-value layer
\begin{equation*}
	d_\text{hvl}=\frac{\log{2}}{b}=\SI{12.8}{\milli\meter}.
\end{equation*}

The literature value\footnote{\url{https://www.nde-ed.org/EducationResources/CommunityCollege/Radiography/Physics/HalfValueLayer.htm}} for gamma radiation from a Co-60 source is $d_\text{hvl,lit}=\SI{0.49}{\inch}$ .
This makes a relative deviation of \SI{9.4}{\percent}, which is more than acceptable.
