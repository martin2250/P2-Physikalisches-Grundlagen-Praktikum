% !TEX root = ../main.tex
\chapter{Absorption of Gamma Particles}
This experiment deals with absorption of gammas emitted by radioisotopes Co-60 and Cs-137.

\section{Absorption in Plumbum}
\begin{figure}[ht!]
	\centering
	\includegraphics[width=0.7\textwidth]{./data/plots/4_1.pdf}
	\caption[Gamma radiation over Pb thickness]{Gamma radiation over Pb thickness}
	\label{fig:gamma_pb}
\end{figure}
To measure the behavior of absorption in plumbum, a block of it is placed between a Co-60 source and the Geiger-Müller tube, graudually increasing the block's thickness.
Once again, background radiation is subtracted from the number of measured events.
Dead time of the tube does not have to be considered, as the tube is operated within the pleateau region.\todo{We'll see that once I've done 1.4.}

\autoref{fig:gamma_pb} depicts the measured data and confirms the \textbf{law of absorption}
\begin{equation*}
	N\propto e^{-b*d}.
\end{equation*}

Fit parameters are
\begin{gather*}
	a=\num{3.76}cps\qquad b=\SI{0.05}{\milli\meter}^{-1}.
\end{gather*}

$b$ can be identified with the coefficient of absorption, which can be plugged into
\begin{equation*}
	\alpha=\frac{b}{\rho_\text{Pb}}=\SI{4.41e-3}{\meter\squared\per\kilogram}
\end{equation*}
to calculate the \todo{mass-coefficient of absorption?}, where $\rho=\SI{11.342}{\gram\per\cubic\centi\meter}$.
Using
\begin{equation*}
	d_\text{hvl}=\frac{\log{2}}{\alpha}.
\end{equation*}
