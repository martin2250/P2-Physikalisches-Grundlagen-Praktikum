% !TEX root = ../main.tex
\chapter{Geiger-Mueller tube characteristics}
\section{Operating voltage}
To determine the optimal operating voltage of the two provided tubes, the count rate is measured for operating voltages between \SI{300}{\volt} and \SI{450}{\volt}, with equal radiation exposure from an $^{90}$Sr beta source.
\autoref{fig:GM-curves} shows the resulting characteristic curves.
Both show the Geiger plateau clearly, however the weak pulses produced by the other operating modes were not recorded by the pulse counter.

The tube used in the left setup shows a slope of \SI{0.11}{\cps\per\volt} in the plateau region, the right tube shows a slope of \SI{0.13}{\cps\per\volt}.
For all further experiments, operating voltages of \SI{450}{\volt} for the left setup, and \SI{380}{\volt} for the right setup were chosen.

\section{Radiation Background}
All artificial sources of radiation are stored behind lead shields in order to determine the natural background radiation level.
To get a realiable rate from such infrequent events, a long gate time of \SI{5}{\second} and a total run time of \todo{...} were used.
The measured values are listed in \autoref{tab:background}.

\begin{table}[b!]\centering
	\caption{Background Radiation Levels}
	\label{tab:background}
	\begin{tabular}{ccc}
		&	mean& standard deviation\\
		left setup&	\SI{0.353}{\cps}&	\SI{0.578}{\cps}\\
		right setup&	\SI{0.324}{\cps}&	\SI{0.559}{\cps}\\
	\end{tabular}
\end{table}

\begin{figure}[b!]
	\centering
	\includegraphics[width=.6\linewidth]{data/plots/1_1.pdf}
	\caption{Characteristic curves of the provided Geiger-Mueller tubes}
	\label{fig:GM-curves}
\end{figure}
