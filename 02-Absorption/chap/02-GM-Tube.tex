% !TEX root = ../main.tex
\chapter{Geiger-Mueller Tube Characteristics}
\section{Operating Voltage}
To determine the optimal operating voltage of the two provided tubes, the count rate is measured for operating voltages between \SI{300}{\volt} and \SI{450}{\volt}, with equal radiation exposure from an $^{90}$Sr beta source.
\autoref{fig:GM-curves} shows the resulting characteristic curves.
Both show the Geiger plateau clearly, however the weak pulses produced by the other operating modes were not recorded by the pulse counter.

The tube used in the left setup shows a slope of \SI{0.11}{\cps\per\volt} in the plateau region, the right tube shows a slope of \SI{0.13}{\cps\per\volt}.
For all further experiments, operating voltages of \SI{450}{\volt} for the left setup, and \SI{380}{\volt} for the right setup were chosen.

\section{Radiation Background}
All artificial sources of radiation are stored behind lead shields in order to determine the natural background radiation level.
To get a realiable rate from such infrequent events, a long gate time of \SI{2}{\second} and a total run time of \SI{340}{\second} were chosen.
The measured values are listed in \autoref{tab:background}.

\begin{table}[b!]\centering
	\caption{Background Radiation Levels}
	\label{tab:background}
	\begin{tabular}{ccc}
		&	mean& standard deviation\\
		left setup&	\SI{0.353}{\cps}&	\SI{0.578}{\cps}\\
		right setup&	\SI{0.324}{\cps}&	\SI{0.559}{\cps}\\
	\end{tabular}
\end{table}

\begin{figure}[b!]
	\centering
	\includegraphics[width=.6\linewidth]{data/plots/1_1.pdf}
	\caption{Characteristic curves of the provided Geiger-Mueller tubes}
	\label{fig:GM-curves}
\end{figure}

\section{Dead Time}
\begin{figure}[tbp]
	\begin{subfigure}{.4\textwidth}
		\centering
		\caption{Event counts for dead time}
		\label{tab:dead_time}
		\begin{tabular}{SSS}
			\toprule
			{$N$ (cps)}	&	{left}&	{right}\\
			\midrule
				{$N_1$}	&	\num{23518}	&	\num{19204}	\\
				{$N_2$}	&	\num{17466}	&	\num{13990}	\\
				{$N_{12}$}	&	\num{42872}	&	\num{34170}	\\
			\bottomrule
		\end{tabular}
	\end{subfigure}
	\begin{subfigure}{.8\textwidth}
		\centering
		\includegraphics[width=.8\linewidth]{data/plots/1_4.pdf}
		\caption{Distance Behavior of Point Sources}
		\label{fig:invsq}
	\end{subfigure}
\end{figure}

As mentioned in \autoref{sec:geiger_shit}, the depletion of electrons prevent other incident particles to induce any gas discharges for a certain amount of time because induced positive ions have to migrate to the cathode (wall of the tube) first for subsequent avalanches to occur.
This time is called \textbf{dead time} $\tau_\text{dead}$.

Dead time can be determined by a two-preparation method using the formula \footnote{Source: Excercise descr.}
\begin{equation}\label{eq:dead_time}
	\tau_\text{dead} = \frac{T}{N_{12}}\left(1-\sqrt{1-\left(N_1+N_2-N_{12}\right)\cdot\frac{N_{12}}{N_1N_2}}\right).
\end{equation}

Utilizing \autoref{eq:dead_time} and the data in \autoref{tab:dead_time} dead time is calculated as
\begin{gather*}
 	\tau_\text{dead,l}=\SI{263.4}{\micro\second}\qquad \tau_\text{dead,r}=\SI{211.6}{\micro\second}
\end{gather*}
for each setup (left-right).

These dead times are needed for corrections of subsequent experiments.
The corrected count rate is
\begin{equation}\label{eq:dead_time_corr}
	R_\text{corr}=\frac{R}{1-R\cdot\tau_\text{dead}}.
\end{equation}

\section{Inverse Square Law}
The validity of the inverse square law is tested by placing a radiation source in front of the tube and measuring the acitivity at various positions.
To minimize the attenuation due to absorption, which does not follow the inverse square law, gamma radiation from a Co-60 is used.

\autoref{fig:invsq} shows the results of the measurements in a log-log-graph.
A linear fit is applied, and the calculted slope is \num{-2.1}.

The inverse square law holds.
