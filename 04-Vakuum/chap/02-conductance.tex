\chapter{Conductance of a Metal Tube}
\section{Experimental Setup}
The corrugated pipe (L) with an inner diameter of $d_\text{co}=\SI{21}{\milli\meter}$ and length of $l_\text{co}=\SI{440}{\milli\meter}$ is replaced with a thinner metal tube of ID $d_\text{m}=\SI{2}{\milli\meter}$ and length $l_\text{m}=\SI{443}{\milli\meter}$.
The rotary vane pump (DP) is turned on and pressure curves of T1 and T2 are recorded.

\section{Theoretical Background}
The rotary vane pump's specified suction capacity $S$ differs from the effective suction capacity $S_\text{eff}$, because of the pipe system that connects recipient and pump.
Like one would intuitively expect, longer pipes lower the effective suction capacity, as they constitute a higher flow resistance for the air.
It holds
\begin{equation}\label{eq:suction_eff}
	\frac{1}{S_\text{eff}}=\frac{1}{S}+\frac{1}{L},
\end{equation}
where $L$ denotes the conductance of the pipe (system).
Rearranging \autoref{eq:suction_eff}, the conductance $L$ can be calculated by
\begin{equation}\label{eq:conductance}
	L=\frac{S\cdot S_\text{eff}}{S-S_\text{eff}}.
\end{equation}
Validity of \autoref{eq:suction_eff} can be affirmed by considering the limit $L\rightarrow\infty$, which results in $S_\text{eff}=S$.
This is consistent with the expectation that an unlowered suction capacity can be expected with no pipe resistance ($L\rightarrow\infty$).
