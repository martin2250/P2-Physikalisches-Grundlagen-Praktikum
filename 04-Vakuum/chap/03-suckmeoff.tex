\chapter{Pressure-dependent Suction Capacity Of a Rotary Vane Pump}
\begin{figure}[hbp]
	\centering
	\includegraphics[width=0.8\textwidth]{./data/plots/3.pdf}
	\caption[Pressure Over Time]{Pressure Curve at T1}
	\label{fig:pressure_shit2}
\end{figure}

\section{Experimental Setup}
The experiment is conducted with the corrugated pipe (L) with an inner diameter of $d_\text{co}=\SI{21}{\milli\meter}$ and length of $l_\text{co}=\SI{440}{\milli\meter}$.
A pressure curve at T1 is recorded to determine the suction capacity as described in \autoref{chap:con}.

\section{Evaluation}
\begin{table}[b!]
	\centering
	\label{tab:values_suc}
	\caption[Fit parameters and suction capacities]{Fit parameter $a$ and resulting suction capacities $S$, $V=\SI{10}{\liter}$}
	\begin{tabular}{S|SSS}
		\toprule
		{}	&	{I}	&	{II}	&	{III}\\
		\midrule
		{$a$ $(\si{\second}^{-1})$}	&	\num{-0.1650}	&	\num{-0.0765}	&	\num{-0.0037}	\\
		{$S$ $(\si{\liter\per\second})$}	&	\num{1.65}	&	\num{0.765}	&	\num{0.037}	\\
		\bottomrule
	\end{tabular}
\end{table}
Lorem ipsum dolor sit amet, consectetur adipisicing elit, sed do eiusmod tempor incididunt ut labore et dolore magna aliqua. Ut enim ad minim veniam, quis nostrud exercitation ullamco laboris nisi ut aliquip ex ea commodo consequat. Duis aute irure dolor in reprehenderit in voluptate velit esse cillum dolore eu fugiat nulla pariatur. Excepteur sint occaecat cupidatat non proident, sunt in culpa qui officia deserunt mollit anim id est laborum.
