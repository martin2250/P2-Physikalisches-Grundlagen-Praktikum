\chapter{Theory}

\section{Pumps}
\subsection{Turbomolecular Pump}

\subsection{Rotary Vane Pump}
The rotary vane pump is a rotational displacement pump.
Inside of the outer casing, the so called stator, an excentrically mounted rotor is rotating.
Vanes are allowed to slide out of this rotor either spring-loadedly or by centrifugal force and are pressed permanently against the inner wall of the pump, effectively creating small chambers which do the pump work.
Lubrication is provided by grease, so the vanes always transport a small amount of lubricant to the outlet as well, which is fed back into the pump afterwards.
Lubrication oil ensures that the individual vane chambers are sealed properly, which increases effectivity.
The pump's suction effect is caused by the increasing volume of the vane chamber as the rotor rotates.
The pumped air is then condensed at the discharge side of the pump and expelled afterwards.
\section{Vakuum Gauges}
\subsection{Ion Gauge}

\subsection{Thermal Conductivity Vacuum Meter}

\section{The Apparatus}
