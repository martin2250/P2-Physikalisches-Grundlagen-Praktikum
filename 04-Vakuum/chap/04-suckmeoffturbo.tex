\chapter{Pressure-dependent Suction Capacity Of a Turbomolecular Pump}
\begin{figure}[hbp]
	\centering
	\includegraphics[width=0.8\textwidth]{./data/plots/4.pdf}
	\caption[Pressure Over Time]{Pressure Curve at IM}
	\label{fig:pressure_shit3}
\end{figure}

\section{Experimental Setup}
The experiment is conducted with the corrugated pipe (L) with an inner diameter of $d_\text{co}=\SI{21}{\milli\meter}$ and length of $l_\text{co}=\SI{440}{\milli\meter}$.
After evacuating the volume to a pressure of around \SI{8e-2}{\milli\bar} with the mechanical pump, the turbomolecular pump is switched on.
Pressure is measured over time with the ion gauge.

\section{Evaluation}
\autoref{fig:pressure_shit3} shows the measured pressure curve.
As was the case in \autoref{chap:3} the measured curve is not linear everywhere.
To fit the linear model, an interval of roughly $t\in[\SI{0}{\second},\SI{30}{\second}]$ is considered.
A slope of $a=\SI{-0.1023}{\per\second}$ is calculated, which makes a suction capacity of \SI{1.023}{\liter\per\second}.
The manufacturer specifies a suction capacity of $\geq\SI{30}{\liter\per\second}$.
This value does not resemble the measured value at all.
A possible explanation is that manufacturers often specify suction capacities at atmospheric pressure.
Consequently, the suction capacity decreases drastically at lower pressures.
Another possible error source could be improper usage of the pump by previous experimenters and a resulting damage at the sensitive pump.
