\chapter{Breakdown Voltage}

The breakdown voltage of air is measured at different pressures.
Two metal ball electrodes are placed in the receiver and connected to a high voltage power supply.
The breakdown voltage is measured by slowly raising the voltage of the power supply and recording the last reading before the breakdown as $U_\text{br}$.
If there is a continuous discharge, the minimal required voltage to sustain the discharge is recorded as $U_\text{hold}$.

The recorded data is shown in \autoref{fig:breakdown}.
There are several missing data points: at high pressures, the power supply is not able to sustain a continuous discharge and at pressures around \SI{1}{\milli\bar}, the breakdown and hold voltage are nearly identical.
At pressures below \SI{10}{\milli\bar} the breakdown occurs at the feed through instead of the electrodes for reasons unknown.

Despite those issues, the curve follows the expectations.
At very low and high pressures, the breakdown voltage is highest, at pressures around \SI{1}{\milli\bar} the curve has a minimum.
As the discharge does not happen at the electrodes, it isn't possible to compare the mean free path to the gap width, the data does not have quantitative value.

\begin{figure}[b!]
	\centering
	\includegraphics[width=.75\textwidth]{data/plots/6.pdf}
	\caption{Breakdown Voltage}
	\label{fig:breakdown}
\end{figure}
