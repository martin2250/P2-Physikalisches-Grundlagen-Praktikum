%!TEX root = ../10-Specific-Heat-Capacity.tex
\chapter{Calorimetry}
The specific heat capacities of aluminium and copper are determined using a calorimeter.

\section{Theory}\label{sec:cal-theory}
To determine the heat capacity of an object, the object and a reference body with a known heat capacity are brought to different temperatures.
The object and water are placed in a calorimeter and the temperatures are left to equalize.
The heat capacity of the object is calculated from the temperatures before and after the equalization process.
In the following experiments water is used as the reference body as it has a well known heat capacity which is approximately constant with temperature and has high thermal conductivity.

The provided calorimeter is a stainless steel (\todo{?}) vacuum flask.
As such, there is little heat transfer to or from the environment.
For near-ambient tempertures and short experiments, the equalization processes can be assumed to be adiabatic
\begin{equation}\label{eq:adiabatic}
	\delta Q = \sum_i m_i c_i \cdot \Delta T_i = 0.
\end{equation}

For all calculations the following variables are used
\begin{itemize}
	\item $m_\text{c}, T_\text{c}$: mass and initial temperature of cold water
	\item $m_\text{h}, T_\text{h}$: mass and initial temperature of hot test object
	\item $T_\text{mix}$: temperature of final thermal equilibrium
	\item $c_\text{w} = \SI{4.187}{\joule\per\gram\per\kelvin}$: specific heat capacity of water
\end{itemize}

\section{Calibration}
Though the calorimeter is well insulated from ambient temperature, it too stores heat with its total heat capacity $C_\text{cal}$.
The amount of heat lost to the calorimeter is determined by filling the calorimeter with room temperature water and adding a roughly equal amount of hot water.
$C_\text{cal}$ varies based on the surface area of the calorimeter that is in contact with the water, so the amount of water is chosen similar to what is used in the other experiments.

Using \autoref{eq:adiabatic} $C_\text{cal}$ is calculated as
\begin{gather*}
	0 = \delta Q = \left(c_\text{w} m_\text{c} + C_\text{cal}\right) \left(T_\text{mix} - T_\text{c}\right) + c_\text{w} m_\text{h} \left(T_\text{mix} - T_\text{h}\right)\\
	\Leftrightarrow C_\text{cal} = c_\text{w} \cdot \left(m_\text{h} \cdot  \frac{T_\text{h} - T_\text{mix}}{T_\text{mix} - T_\text{c}} - m_\text{c}\right).
\end{gather*}
Substituting the experimentally determined values $m_\text{c} = \SI{50.9}{\gram}, m_\text{h} = \SI{39.9}{\gram}, T_\text{c} = \SI{27.7}{\celsius}, T_\text{h} = \SI{88}{\celsius} \text{ and } T_\text{mix} = \SI{51.4}{\celsius}$ gives $C_\text{cal} = \SI{44.25}{\joule\per\kelvin}$.
This value is used in the calculation of the specific heat capacities.

\section{Evaluation}
The heat capacities of each two different size cylinders of aluminium and copper are determined as described in \autoref{sec:cal-theory}.
Again using \autoref{eq:adiabatic} the specific heat capacity of the tested material is calculated as
\begin{gather*}
	0 = \delta Q = \left(c_\text{w} m_\text{c} + C_\text{cal}\right) \left(T_\text{mix} - T_\text{c}\right) + c_\text{spec} m_\text{h} \left(T_\text{mix} - T_\text{h}\right)\\
	\Leftrightarrow c_\text{spec} = \frac{\left(c_\text{w} m_\text{c} + C_\text{cal}\right) \left(T_\text{mix} - T_\text{c}\right)}{m_\text{h} \left(T_\text{h} - T_\text{mix}\right)}
\end{gather*}
The experimental values are shown in \autoref{tab:fancy-shit} along with calulated specific heat capacities and deviations from the literature values $c_\text{Al} = \SI{0.897}{\joule\per\gram\per\kelvin}$ and $c_\text{Cu} = \SI{0.385}{\joule\per\gram\per\kelvin}$.

It is worth noting that all results lie above the literature values, so there is a systematic error.
A possible cause could be that the temperature of the water bath, which is used to heat up the samples, drops more quickly than the temperature of the sample itself.
This would lead to a lower reading for $T_\text{h}$ and to a higher apparent specific heat capacity.
Another error source is the heat capacity of the calorimeter which is dependent on it's fill level.

Considering the small sample sizes which make the experiment more sensitive to losses and unaccounted for heat capacities, the results are adequate.



\begin{table}[tbp]
	\centering
	\caption[Calorimeter Results]{\textbf{Calorimeter Results}, water and the examined material (masses $m_\text{c}$ and $m_\text{h}$, temperatures $T_\text{c}$ and $T_\text{h}$) are added to calorimeter (temperature $T_\text{c}$), resulting in equilibrium temperature $T_\text{mix}$. $c_\text{spec}$ with relative deviation from literature value is the calculated specific heat capacity, corrected for calorimeter's own heat capacity.}
	\label{tab:fancy-shit}
	% \begin{tabular}{lSSSSSSS[retain-explicit-plus]}
	\begin{tabular}{lSSSSSSr}
	\toprule
	{material}&
	{$m_\text{c}$ (\si{\gram})}&
	{$m_\text{h}$ (\si{\gram})}&
	{$T_\text{c}$ (\si{\celsius})}&
	{$T_\text{h}$ (\si{\celsius})}&
	{$T_\text{mix}$ (\si{\celsius})}&
	{$c_\text{spec}$ (\si{\joule\per\gram\per\kelvin})}&
	% {rel. deviation (\si{\percent})}\\
	{rel. deviation}\\
	\midrule
		\input{data/plots/calorimeter.agtex}
	\bottomrule
	\end{tabular}
\end{table}
