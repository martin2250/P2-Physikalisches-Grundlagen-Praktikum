%!TEX root = ../10-Specific-Heat-Capacity.tex
\chapter{Temperature-dependent, Specific Heat Capacity of Aluminium}
\section{Setup}
A solid aluminium block \todo{which oddly looks like a hollow cylinder} is cooled down to a temperature of roughly \SI{-196}{\celsius} using liquid nitrogen.
The heating process then takes place in two different ways:	\todo{doesn't just take place by itself (well sorta), better is carried out}
\begin{itemize}
	\item \textbf{Without heater} This measurement extends over a long period of time ($\approx$ one day). \todo{maybe mention imperfect insulation -> ambient heat}
	\item \textbf{With heater} Employing a heat input $P_\text{el}$.	\todo{didn't see a single mention about what kind of heater is used}
\end{itemize}
The sample's temperature is measured over time using a thermocouple.

\section{Theory}
Using \autoref{eq:adiabatic}, the heating process with heating power $P_\text{el}$ can be described as
\begin{alignat}{3}
										&\delta Q 																	&&= m\cdot c(T)\cdot\d T \nonumber \\
	\Leftrightarrow\ 	&P\d t																			&&= m\cdot c(T)\cdot\d T \nonumber \\
	\Leftrightarrow\ 	&\left(\frac{\d T}{\d t}\right)_\text{tot}	&&= \frac{P_\text{env}+P_\text{el}}{mc}, \label{eq:diff_temp}
\end{alignat}
where $P_\text{env}$ denotes the heating power applied by the environment.	\todo{why split this sentence up?}

Rearranging \autoref{eq:diff_temp} yields
\begin{alignat}{3}
										& \frac{P_\text{el}}{mc}	&&=\dot{T}_\text{tot}-\dot{T}_\text{env} \nonumber \\
	\Leftrightarrow\	& c(T)										&&=\frac{P_\text{el}}{m\left(\dot{T}_\text{tot}(T)-\dot{T}_\text{env}(T)\right)}, \label{eq:cap_temp}
\end{alignat}
where $\dot{T}_\text{tot}$ and $\dot{T}_\text{env}$ denote the heating curves' gradients with additional heat input by the heater and without heat input respectively.

\section{Evaluation}
\begin{figure}[tbp]
	\centering
	\begin{subfigure}{.45\textwidth}
		\centering
		\includegraphics[width=1.1\textwidth]{./data/plots/al_heater.pdf}
		\caption[Heating curve with heater]{\textbf{Heating curve with heater}}
		\label{fig:heater}
	\end{subfigure}
	\quad
	\begin{subfigure}{.45\textwidth}
		\centering
		\includegraphics[width=1.1\textwidth]{./data/plots/al_no_heater.pdf}
		\caption[Heating curve without heater]{\textbf{Heating curve without heater}}
		\label{fig:no_heater}
	\end{subfigure}
	\caption[Heating curves for aluminium]{Fits for the function $T(t)=a\cdot t^b + c$ are applied. Fit parameters are shown in their respecive legends.}
\end{figure}

A standard type-K thermocouple is used, the sample's temperature is recovered from it's output voltage through an interpolated conversion table.
\todo{mention shitty as f*** method of using an ice bath as reference junction}
% To calculate the temperature using the thermocouple's output voltage, a table with conversion		% this is all standard stuff, nobody cares values\footnote{\url{/www-ekp.physik.uni-karlsruhe.de/~simonis/praktikum/p2/p2-versuchsanleitungen/Waermekapazitaet.pdf}} is used.
% Using a python script, a polynomial fit is applied to this conversion table to define a function which outputs a temperature value for any desired voltage.
Plotting the thus obtained temperatures over time yields figures \ref{fig:heater} and \ref{fig:no_heater}.	\todo{did you even read julia's email? plotting is colloquial}

The fit function $T(t)=a\cdot t^b + c$ is applied to both curves.	%guessing that sentence is going to change anyways
Inverting this function yields
\begin{equation}\label{eq:inv}
	t(T) = \left(\frac{T-c}{a}\right)^\frac{1}{b}.
\end{equation}
Differentiation of the fit function gives
\begin{equation}\label{eq:diff}
	\dot{T}(t) = ab\cdot t^{b-1}.
\end{equation}
Insertion of \autoref{eq:inv} into \autoref{eq:diff} finally leads to the expression
\begin{equation}\label{eq:T_dot_of_T}
	\dot{T}(T) = ab\cdot\left(\frac{T-c}{a}\right)^{\frac{b-1}{b}},
\end{equation}
which can be used to compute the quantaties $\dot{T}_\text{tot}$ and $\dot{T}_\text{env}$ necessary for \autoref{eq:cap_temp}.

Considering the fit parameters shown in their respective figures, the functions
\begin{equation*}
	\dot{T}_\text{tot} = 0.7102\cdot\left(\frac{T+197}{1.06}\right)^{\frac{0.67-1}{0.67}}
\end{equation*}
and
\begin{equation*}
	\dot{T}_\text{env} = 27.84\cdot\left(\frac{T+299.28}{27.62}\right)^{\frac{0.22-1}{0.22}}
\end{equation*}
\todo{verb much?}.

\begin{figure}[tbp]
	\centering
	\includegraphics[width=0.6\textwidth]{./data/plots/cap_over_T.pdf}
	\caption[Temperature dependency of aluminium's specific heat capacity]{\textbf{Temperature dependency of aluminium's specific heat capacity} $c(\SI{20}{\celsius})=\SI{1.3}{\joule\per\gram\kelvin}$ at $P_\text{el}=\SI{22.31}{\watt}$ and $m=\SI{338}{\gram}$}
	\label{fig:cap}
\end{figure}
Inserting these functions into \autoref{eq:cap_temp} and plotting \todo{colloquial} the heat capacity against the temperature yields \autoref{fig:cap}.
A heating power \todo{how known?} of $P_\text{el}=\SI{22.31}{\watt}$ and a sample mass of $m=\SI{338}{\gram}$ is used.
It is easy to see that the heat capacity curve is approximately linear for temperatures above \SI{-100}{\celsius}.
The experimentally determined specific heat capacity at \SI{20}{\celsius} is equal to \SI{1.3}{\joule\per\gram\kelvin} which relatively deviates by \num{45.2}\% from the literature value \footnote{\url{en.wikipedia.org/wiki/Heat_capacity}} of \SI{0.895}{\joule\per\gram\kelvin}.
This relatively large deviation could be reduced by finding an invertible function that fits the curve seen in \autoref{fig:no_heater} better \todo{well, go ahead}.
